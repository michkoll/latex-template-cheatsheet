\raggedright
\footnotesize
\begin{multicols}{3}	
	\setlength{\premulticols}{1pt}
	\setlength{\postmulticols}{1pt}
	\setlength{\multicolsep}{1pt}
	\setlength{\columnsep}{2pt}

\begin{center}
     \Large{\underline{Netzwerkforensik}} \\
\end{center}

\section{Sniffing}
\section{Tools}
\settowidth{\MyLen}{\texttt{option.2.spa}}
\begin{tabular}{@{}p{\the\MyLen}
		@{}p{\linewidth-\the\MyLen}@{}}
	\texttt{cURL} & Einfaches Programm zum Senden von Netzwerk-Requests. Unterstützte Protokolle sind unter anderem HTTP, HTTPS, FTP und FTPS.\\
	\texttt{dig} & Befehl zum Abfragen des Domain Name Systems (Alternative zu nslookup).\\
	\texttt{dsniff} & Tools zum Sniffen von Passwörtern und Analysieren von Netzwerkdatenverkehr allgemein.\\
	\texttt{Ettercap} & Tool zum Durchführen von Man-in-the-middle-Angriffen, beispielsweise mittels ARP-Spoofing.\\
	\texttt{filesnarf} & Dateisniffer für NFS-Datenverkehr. (In dsniff enthalten.)\\
	\texttt{mailsnarf} & Sniffer für Mails im Berkeley mbox format. (In dsniff enthalten.)\\
	\texttt{msgsnarf} & Sniffer für ältere bekannte Chat-Messenger (ICQ, IRC, MSN Messenger usw.)\\
	\texttt{nmap} & Etablierter Konsolen-basierter Portscanner.\\
	\texttt{Scapy} & Tool zum Manipulieren von Paketen im Netzwerkverkehr.\\
	\texttt{urlsnarf} & Sniffer für HTTP-Requests. (In dsniff enthalten.)\\
	\texttt{pcap} & API für Sniffer, die von Tools wie Tcpdump, nmap usw. verwendet wird.\\
	\texttt{Tcpdump} & Bekannter und verbreiteter Paketsniffer (Kommandozeilentool).\\
	\texttt{Wireshark} & Etablierter Netzwerksniffer für Pakete verschiedener Protokolle\\
\end{tabular}
\end{multicols}