\raggedright
\footnotesize
\begin{multicols}{3}	
	\setlength{\premulticols}{1pt}
	\setlength{\postmulticols}{1pt}
	\setlength{\multicolsep}{1pt}
	\setlength{\columnsep}{2pt}

\begin{center}
     \Large{\underline{Reverse-Engineering}} \\
\end{center}

\section{Tools}
\settowidth{\MyLen}{\texttt{option.2.spa}}
\begin{tabular}{@{}p{\the\MyLen}
		@{}p{\linewidth-\the\MyLen}@{}}
	\texttt{IDA} &  Vollständiger Name: Interactive Disassembler. Von Microsoft entwickelter Disassembler, der Skripting erlaubt. \\
	\texttt{ildasm} & Einfacher GUI-basierter Disassembler für PE-Anwendungen, die IL-Code enthalten.\\
	\texttt{OllyDbg} & Etablierter Debugger für 32-Bit Anwendungen auf Windows.\\
	\texttt{WinDbg} & Debugger für Windows Kernel- und Usermode, der die Analyse von crash dumps und CPU-Register erlaubt.\\
\end{tabular}
\lipsum
\section{Verhinderung von Disassemblierung}
\lipsum
\section{Obfuscation}
Obfuscation bezeichnet allgemein eine Veränderung des Programmcodes, um die Lesbarkeit bzw. das Reverse Engineering des Programms zu erschweren. In den folgenden Subsections werden einige Techniken beschrieben.
\section{Function-Splitting}
Beim Function-Splitting wird eine Funktion f \enquote{kopiert} (im Folgenden f' genannt) (und dann im Idealfall an einer ganz anderen Stelle im Programm abgelegt und inhaltlich möglichst weiter obfuscatet, damit man möglichst schwer erkennen kann, dass die beiden Funktionen inhaltlich das gleiche machen). Wenn im bisherigen Programm f von 2 Stellen (im Folgenden g1 und g2 genannt) aus aufgerufen wird, dann wird der Programmcode prinzipiell dahingehend angepasst, dass g1 f aufruft und g2 f' aufruft. Es ist dadurch schwerer erkennbar, dass an dieser Stelle g1 und g2 inhaltlich die gleiche Funktion ausführen.

\subsection{Junk-Code}
Junk-Code bezeichnet Programmcode, der zur korrekten Programmausführung nicht erforderlich ist. Er dient lediglich dazu, einem Reverse-Engineerer mehr Arbeit zu machen, da es nicht immer leicht erkennbar ist, ob Code Junk-Code ist oder nicht.
\subsection{Fake-Loops}
Als Fake-Loops werden Loops (for-Loops, while-Loops, etc.) bezeichnet, die den Anschein erwecken sollen, dass der Schleifeninhalt öfters ausgeführt wird. In Wirklichkeit wird der Inhalt der Schleife jedoch nur einmal oder womöglich auch gar nicht ausgeführt (z. B. wenn sie ausschließlich mit Junk-Code gefüllt ist).
\section{Decompilierung}
\lipsum
\end{multicols}